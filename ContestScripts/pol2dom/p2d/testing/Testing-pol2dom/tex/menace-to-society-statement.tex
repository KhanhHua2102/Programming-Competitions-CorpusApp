\documentclass[11pt,final,twoside]{article}

%%%%%%%%%%%%%%%%%%% PACKAGES %%%%%%%%%%%%%%%%%%%%
\usepackage[a4paper, top=2.5cm, bottom=3.5cm, inner=2cm, outer=2.5cm]{geometry}

\usepackage[usenames,dvipsnames]{xcolor}
\usepackage[T1]{fontenc}    
\usepackage[utf8]{inputenc}
\usepackage{verbatim}
\usepackage{enumitem}
\usepackage{layout}
\usepackage{fancyhdr}
\usepackage{ifthen}
\usepackage{graphicx}
\usepackage{pdfpages}
\usepackage{calc}
\usepackage{amsmath}
\usepackage{amssymb}
\usepackage{hyperref}
\hypersetup{
    colorlinks=true,
    urlcolor=blue
}
\usepackage{tikz}
\usepackage[strict]{changepage}
\usepackage{xparse}


% Set max width of \includegraphics to \linewidth.
\usepackage[export]{adjustbox}
\let\oldincludegraphics\includegraphics
\renewcommand{\includegraphics}[2][]{%
    \oldincludegraphics[#1,max width=\linewidth]{#2}%
}

%%%%%%%%%%%%%%%%%%% GLOBAL SPACING CONFIGURATION %%%%%%%%%%%%%%%%%%%%
\setlist{leftmargin=2em,topsep=0.5em}
\setlength{\parindent}{0em}
\setlength{\parskip}{0.5em}
\raggedbottom

\usepackage{titlesec}
\titlespacing*{\section}{0em}{1.5em}{0.2em}
\titleformat*{\section}{\Large\scshape\bfseries}

\titlespacing*{\subsection}{0em}{-1.5em}{-0.4em}
\titleformat*{\subsection}{\normalsize\bfseries}

%%%%%%%%%%%%%%%%%%% HEADER AND FOOTER %%%%%%%%%%%%%%%%%%%%

\fancypagestyle{problem}{
\fancyhf{}
    \renewcommand{\headrulewidth}{0.1pt}
\setlength\headheight{14pt}
\lhead{\textsc{Problem \problemlabel: \problemtitle}}
\rhead{\textsc{\contestname}} % e.g., SWERC 2021/22 - Milan
\rfoot{\thepage\hspace{2em}}
}

\fancypagestyle{solution}{
\fancyhf{}
    \renewcommand{\headrulewidth}{0.1pt}
\setlength\headheight{14pt}
\lhead{\textsc{\problemlabel: \problemtitle}}
\rhead{\textsc{Solutions of \contestname}} % e.g., SWERC 2021/22 - Milan
\rfoot{\thepage\hspace{2em}}
}


%%%%%%%%%%%%%%%%%%%%%%% BLANK PAGES %%%%%%%%%%%%%%%%%%%%%%%

\fancypagestyle{blank}{
    \fancyhf{}
    \renewcommand{\headrulewidth}{0pt}    
    \rfoot{\thepage\hspace{2em}}
}

\newcommand{\insertblankpageifnecessary}{
    \clearpage
    \checkoddpage
    \ifoddpage\else
    \thispagestyle{blank}
    \vspace*{\fill}
    \begin{center}
    \scalebox{3}{\rotatebox{45}{\color{black!6}\Huge\textbf{BLANK PAGE}}}
    \vspace{80pt}
    \end{center}
    \vspace*{\fill}
    \fi
    \cleardoublepage
}

%%%%%%%%%%%%%%%%%%% PROBLEM TITLE %%%%%%%%%%%%%%%%%%%%

\newcommand\balloon{%
\if \showballoon 1
\begin{tikzpicture}[scale=0.5, overlay, shift={(34.5, 0.5)}]
    \shade[ball color = \problemcolorname] ellipse (1.75 and 2);
    \shade[ball color = \problemcolorname] (-.1,-2) -- (-.3,-2.2) -- (.3,-2.2) -- (.1,-2) -- cycle;
    \path (0, -2.2) edge [out=250, in=120] (0.3, -4);
    \path (0.3, -4) edge [out=-60, in=60] (0, -6);
\end{tikzpicture}
\fi
}

\newcommand\tlml{%
\if \showtlml 1
\begin{flushright}
    \begin{minipage}[t]{4.5cm}
        \textsc{Time limit: \hspace{1.55em}\timelimit{}s} \\
        \textsc{Memory limit: \memorylimit{}MB}
        % The memory limit is in MiB, but most contestants don't know the difference and the difference is minimal, so we prefer to write MB.
    \end{minipage}
\end{flushright}
\fi
}

\newcommand\problemheader{%
\setcounter{samplescnt}{0}
\balloon
{\bf \huge \fbox{\textsc{\problemlabel}} \problemtitle}
\tlml
\vspace{2em}%
}

\newcommand\solutionheader{%
{\bf \huge \fbox{\textsc{\problemlabel}} \problemtitle}
\begin{flushright}
    \begin{tabular}{l l}
        \textsc{Author:} & \textsc{\problemauthor{}} \\ 
        \textsc{Preparation:} & \textsc{\problempreparation{}}     
    \end{tabular}
\end{flushright}
\vspace{1em}%
}

%%%%%%%%%%%%%%%%%%% SAMPLES PRETTY PRINTING %%%%%%%%%%%%%%%%%%%%
\newcounter{samplescnt}

\newcommand\printfile[2]{%
\begin{minipage}[t]{#1}
\vspace{-0.1em}
{\verbatiminput{#2} }
\vspace{-0.5em}
\end{minipage}%
\ignorespacesafterend
}

\newcommand\sampleexplanation[1]{
\subsection*{Explanation of sample \arabic{samplescnt}.}
#1%

\addvspace{2em}
}%


%%%%%%%%%%%%%%%%%%% SMALL SAMPLE %%%%%%%%%%%%%%%%%%%%
\newlength\smallsamplewidth
\setlength\smallsamplewidth{8.08cm}

\newcommand\smallsample[1]{
\stepcounter{samplescnt}%
\begin{tabular}{| c | c |}
    \hline
    \textbf{Sample input \arabic{samplescnt}} & \textbf{Sample output \arabic{samplescnt}} \\
    \hline
    \printfile{\smallsamplewidth}{#1.in}
    &
    \printfile{\smallsamplewidth}{#1.out}
    \\
    \hline
\end{tabular}%

\addvspace{2em}
\ignorespacesafterend
}


%%%%%%%%%%%%%%%%%%% BIG SAMPLE %%%%%%%%%%%%%%%%%%%%
\newlength\bigsamplewidth
\setlength\bigsamplewidth{16.58cm}

\newcommand\bigsample[1]{
\stepcounter{samplescnt}%
\begin{tabular}{| c |}
    \hline
    \textbf{Sample input \arabic{samplescnt}} \\
    \hline
    \printfile{\bigsamplewidth}{#1.in}
    \\
    \hline
\end{tabular}%
\\[1em]
\begin{tabular}{| c |}
    \hline
    \textbf{Sample output \arabic{samplescnt}} \\
    \hline
    \printfile{\bigsamplewidth}{#1.out} 
    \\
    \hline
\end{tabular}%

\addvspace{2em}
\ignorespacesafterend
}

%%%%%%%%%%%%%%%%%%%%%%%% SAMPLE %%%%%%%%%%%%%%%%%%%%%%%%%
% This magic trick to capture the shell output was copied from
% tex.stackexchange.com/questions/16790
\ExplSyntaxOn
\NewDocumentCommand{\captureshell}{som}
 {
  \sdaau_captureshell:Ne \l__sdaau_captureshell_out_tl { #3 }
  \IfBooleanT { #1 }
   {% we may need to stringify the result
    \tl_set:Nx \l__sdaau_captureshell_out_tl
     { \tl_to_str:N \l__sdaau_captureshell_out_tl }
   }
  \IfNoValueTF { #2 }
   {
    \tl_use:N \l__sdaau_captureshell_out_tl
   }
   {
    \tl_set_eq:NN #2 \l__sdaau_captureshell_out_tl
   }
 }

\tl_new:N \l__sdaau_captureshell_out_tl

\cs_new_protected:Nn \sdaau_captureshell:Nn
 {
  \sys_get_shell:nnN { #2 } { } #1
  \tl_trim_spaces:N #1 % remove leading and trailing spaces
 }
\cs_generate_variant:Nn \sdaau_captureshell:Nn { Ne }
\ExplSyntaxOff

\newcommand\sample[1]{
    \captureshell*[\linelengthin]{cat #1.in | wc -L}
    \captureshell*[\linelengthout]{cat #1.out | wc -L}
    \ifnum \linelengthin>40
        \bigsample{#1}
    \else
        \ifnum \linelengthout>40
            \bigsample{#1}
        \else
            \smallsample{#1}
        \fi
    \fi
}

%%%%%%%%%%%%%%%%%%% CONTEST METADATA %%%%%%%%%%%%%%%%%%%%
\newcommand\contestname{Testing-pol2dom}
\newcommand\showballoon{1}
\newcommand\showtlml{1}

%%%%%%%%%%%%%%%%%%% PROBLEM METADATA %%%%%%%%%%%%%%%%%%%%

\newcommand\problemlabel{undefined}
\newcommand\problemcolor{undefined}
\newcommand\problemcolorname{undefined}
\newcommand\problemtitle{undefined}
\newcommand\timelimit{undefined}
\newcommand\memorylimit{undefined}
\newcommand\problemauthor{undefined}
\newcommand\problempreparation{undefined}

 
\begin{document}

\renewcommand\problemlabel{A}
\renewcommand\problemcolor{6495ED}
\renewcommand\problemtitle{A Menace 2 \$ociety}
\renewcommand\timelimit{15.0}
\renewcommand\memorylimit{2048}

\renewcommand\problemcolorname{problemcolorname\problemlabel}
\definecolor\problemcolorname{HTML}{\problemcolor}

\pagestyle{problem}

\problemheader

Jay and Jay's son ja\$on are playing a game together. This game involves counting
occurrences of a string $J$ in another string $S$. But, ja\$on has
gotten so good at this game, Jay is at a loss for how to keep it interesting.

Jay is a menace to society and has decided to make the game more fun for Jay's
son ja\$on by adding some new rules. On top of $S$, Jay will define $K + 1$ more strings
$S_0, S_1, \dots, S_K$. $S_i$ can be created by taking the first character in $S$, skipping the next $i$ characters, then taking the next character, then skipping the next $i$ characters, and so on. For example, if $S$ is \texttt{southpacific}, then $S_2$ is \texttt{staf}. Note that $S_0$ and $S$ are the same string.

Further, in honour of Jay's son ja\$on's name, Jay will use letters from a language called `Engli\$h', which contains all lowercase English letters, except that the letter \texttt{s} is replaced with the letter \texttt{\$}. So \texttt{\$outh}, \texttt{pacific}, \texttt{programming}, \texttt{conte\$t} are valid words in Engli\$h, but \texttt{south} and \texttt{contest} are not.

In Jay's new game, he must find the string $J$ in each of the $K+1$ words.
For example, if $S$ is \texttt{aacbc} and $J$ is \texttt{ac}, then $S_1$ is \texttt{acc}, and $S_2$ is \texttt{ab}. $J$ is found once in both $S_0$ and $S_1$, but not $S_2$.

Given the strings $S$ and $J$, as well as the value of $K$, determine the number of occurrences of $J$ in each $S_i$.


\section*{Input}
The first line of the input contains a single integer $K$~($0 \leq K < 1\,000\,000$),
which is the number of strings to check.

The second line of the input contains a single non-empty string $S$ that has strictly more than $K$ and at most $1\,000\,000$ characters, consisting of Engli\$h letters.

The third line of the input contains a single non-empty string $J$, consisting of Engli\$h letters. The length of $J$ is at most the length of $S$.


\section*{Output}
Display the number of occurrences of $J$ in $S_0, S_1, \dots, S_K$.




\section*{Samples}
\sample{/Users/vybien/Desktop/Projects/cits3200-project/ContestScripts/pol2dom/p2d/testing/Testing-pol2dom/tex/samples/menace-to-society-1}
\sample{/Users/vybien/Desktop/Projects/cits3200-project/ContestScripts/pol2dom/p2d/testing/Testing-pol2dom/tex/samples/menace-to-society-2}
\sample{/Users/vybien/Desktop/Projects/cits3200-project/ContestScripts/pol2dom/p2d/testing/Testing-pol2dom/tex/samples/menace-to-society-3}
\sample{/Users/vybien/Desktop/Projects/cits3200-project/ContestScripts/pol2dom/p2d/testing/Testing-pol2dom/tex/samples/menace-to-society-4}



\end{document}
