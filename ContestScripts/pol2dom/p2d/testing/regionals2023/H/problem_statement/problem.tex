\problemname{Hiring an Accountant
}

Jay's son ja\$on is looking for an accountant to help him with his finances
after selling OTM call options on NVDA. He has a list of $N$ accountants which
he interviews one by one, in order. After intreviewing the $i$th accountant he either
offers the accountant a job or not. If he offers the accountant a job, the
accountant will either accept or reject the offer. If the accountant accepts the
offer, Jay's son ja\$on will hire the accountant and will stop interviewing accountants. If the accountant rejects the
offer, Jay's son ja\$on will move on. However, he will miss the next $M_i$
accountants and not be able to interview them due to wasted time on the rejected
offer.

Each accountant has a skill level $S_i$ and a probability $C_i$ of accepting the
offer. Jay's son ja\$on wants to maximise the expected skill level
of the accountant he hires because he minimised the value of his brokerage
account.

\begin{center}
    \includegraphics[width=0.4\textwidth]{fig}
\end{center}

Jay's son ja\$on has asked you to come up with a strategy to maximise the expected skill
level of the accountant he hires. The strategy you provide should tell him, for each
accountant on the list to be interviewed, whether or not to offer them a job if he
has a chance to (that is if all earlier candidates were not offered jobs or rejected the offers).
Note that this means there are $2^N$ unique strategies you could provide Jay's son ja\$on with.
The expected skill level hired for a strategy is the average over a very large number of trials
(e.g., $10^{1000}$). Each trial is an independent application of the strategy with the accountants making
their weighted random choices.

If Jay's son ja\$on doesn't get an accountant, he
will have to do it himself with his skill of 0.

Jay's son ja\$on doesn't want to have to change his name back to Jay's son
jason. So, please help Jay's son ja\$on maximise the expected skill
level of his accountant!


\section*{Input}

The first line of the input contains a single integer $N$~($1 \leq N \le 200\,000$),
denoting the number of accountants to consider.

The second line contains $N$ integers $S_1, S_2, \ldots, S_N$~($1 \leq S_i \leq 10^6$),
each denoting the skill of the $i$th accountant.

The third line contains $N$ floating point numbers $C_1, C_2, \ldots, C_N$~($0.0
\leq C_i \leq 1.0$), each denoting the probability that the $i$th accountant will accept Jay's son ja\$on's
offer. Each floating point number has at most 6 digits after the decimal point.

The fourth line contains $N$ integers $M_1, M_2, \ldots, M_N$~($0 \leq M_i \leq N$),
denoting how many accountants Jay's son ja\$on will miss the opportunity to
interview if the $i$th accountant rejects an offer. It is possible that $i+M_i+1 > N$.


\section*{Output}

Display the maximum achievable expected skill level of the hired accountant to within an absolute or relative error of $10^{-6}$.
